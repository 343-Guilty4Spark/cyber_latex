\chapter{User Manual for Malware Honeypot}
\label{usermanual}
In this User manual it is given a brief explanation of how to raise up the cluster and launch the attack. For this Project python is requested, if it is already installed skipp the following section.
\section{How to Install python}
For this project is requested a version of python up to 3.X.X. To install python and the used package in the environment of each Raspberry Pi follow this steps:
\begin{enumerate}
\item Install the required packages for python with this command  \begin{verbatim}\$ sudo apt install build-essential zlib1g-dev libncurses5-
dev libgdbm-dev
 libnss3-dev libssl-dev libreadline-dev libffi-dev wget\end{verbatim}
\item Install python (in this case 3.8) \begin{verbatim}\$ sudo apt install python3.8\end{verbatim} 
To check if the correct version of python is installed the following command can be inserted: \begin{verbatim}\$ python --version\end{verbatim}
\item Install pip3 package with the command  \begin{verbatim}\$ sudo apt install python3-pip\end{verbatim} 
\item Install the PAHO MQTT library with the command  \begin{verbatim}\$ pip3 install paho-mqtt\end{verbatim} 
\end{enumerate}
 \section{How to set the cluster environment}
After connecting throw ssh with the raspberry Pi and cloning the repository on each Raspberry Pi follow this passage for each of them:
\begin{enumerate}
\item Enter in the folder "fs\_creato" with the command  \begin{verbatim}\$ cd DELIVERY/NODE_DELIVERY/fs_creato\end{verbatim} 
\item Type the following two commands, the first is needed to specify the interpreter for the bash script while the other creates the file system \begin{verbatim} ~/DELIVERY/NODE_DELIVERY/fs_creato $ sed -i -e 's/\r$//' script2.sh  \end{verbatim}
\begin{verbatim} ~/DELIVERY/NODE_DELIVERY/fs_creato $ chmod +777 ./script2.sh \end{verbatim}  
At the end It has to be displayed in the terminal of the Raspberry Pi \begin{verbatim} pi<number_of_the_pi> created
subscribed to PoliTo/C4ES/# \end{verbatim}  
\end{enumerate}
\section{How to lauch the malware attack}
After cloning the repository in the device is used for this purpose follow this steps:
\begin{enumerate}
\item Enter in the folder "MALWARE\_DELIVERY" with the command  \begin{verbatim}\$ cd DELIVERY/MALWARE_DELIVERY\end{verbatim} 
\item Type the following two commands again  the first is needed to specify the interpreter for the bash script while the other creates the malware  \begin{verbatim} ~/DELIVERY/MALWARE_DELIVERY $ sed -i -e 's/\r$//' script.sh \end{verbatim}
\begin{verbatim} ~/DELIVERY/MALWARE_DELIVERY $ chmod +777 ./script.sh \end{verbatim}  
At the end It has to be displayed in the terminal of the Raspberry Pi \begin{verbatim} MALWARE created
subscribed to PoliTo/C4ES/# \end{verbatim}  

And it starts to inject the public key into the File system after a while a message \begin{verbatim} operation completed successfully \end{verbatim}  has to be shown
When the encryption starts from the malware a message is shown in the terminal of the Raspberry Pi attacked regarding a mismatch of the sentinel file, the procedure is shut down and the connection closed.
\begin{verbatim} Start shutdown procedure RPI...
blacklist cleared
Connection to <IP> closed by remote host.
Connection to <IP> closed. \end{verbatim}
It can be also checked the insertion inside the blacklist in all the other Raspberry Pi just checking the terminal connected (via ssh) to the others Raspberry Pi.
\end{enumerate}
\section{Plus: How to check how many files have been encrypted}
It is a feature provided to check the effective number of encrypted file.
\begin{enumerate}
\item Enter in the folder "NODE\_DELIVERY" with the command  \begin{verbatim}\$ cd DELIVERY/NODE_DELIVERY\end{verbatim} 
\item Insert the command \begin{verbatim}\$  ~/DELIVERY/NODE_DELIVERY $ python3 ./count.py\end{verbatim} 
to check how many files have been encrypted and which one.
\end{enumerate}
\section{Plus: How to visualize the log file}
\begin{enumerate}
\item Enter in the folder "fs\_creato" with the command  \begin{verbatim}\$ cd DELIVERY/NODE_DELIVERY/fs_creato\end{verbatim}
\item Insert the command \begin{verbatim}\$  ~/DELIVERY/NODE_DELIVERY/fs_creato $ less log.txt\end{verbatim} 
It is possible to see the name of the node that perform the encryption, the public key and the steps it performs.
\end{enumerate}



