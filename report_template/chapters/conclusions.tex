\chapter{Conclusions}
In this project we made a survey of honeypot categories and implementations as for the current state of the art. We developed then two solutions targeting malware and DoS attacks respectively. For the first one we implemented an IoT cluster based on rasberry pi systems exploting MQTT communication protocol to send and receive bash commands. We showed how malwares, ransomware specifically, could interact in such a system and proposed some common countermeasures to the problem. Results of the honeypot research have been stored in specific log files that could be then used to update the system. We believe this approach to be good in discovering, then targeting, new ransomware attacks thus improving existing countermearuses. For the second proposal we implemented a tree cluster able to protect the system from DoS attack thanks to a dispatcher which protects the server first creating different socket for different client and managing their packets and then by filtering unwanted packets by sending them to the honeypot which checks that the ip is not blacklisted and that it is sending sane packets, if not it will blacklist it and mark its connection as untrasted.
