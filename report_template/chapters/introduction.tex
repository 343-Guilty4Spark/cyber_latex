\chapter{Introduction}
This document aims at providing an overview of the state-of-the-art regarding honeypots and provide an implementation of Honeypot inside a cluster IoT.\\
Due to the fact that exists various type of honeypot,  two solutions has been presented for different type of cluster and attack. One solution for a honeypot in a (tipologia?) cluster that protects from Malware attacks and the other for a tree structured cluster that protects from DoS attacks. 
(Da modificare alla fine)
The first chapter shows the criteria according to which you can classify honeypots; the second chapter gives a closer look to some of the possible honeypot typologies and the third chapter described the two use cases; an appendix with useful links is present in the last chapter.
% DELETE THE TEXT BELOW
\\\\
In this first chapter we expect you to introduce the project explaining what the project is about, what is the final goal, what are the topics tackled by the project, etc.\newline The introduction must not include any low-level detail about the project, avoid sentences written like: we did this, then this, then this, etc.\newline It is strongly suggested to avoid expressions like `We think`, `We did`, etc...it is better to use impersonal expressions such as: `It is clear that`, `It is possible that`, `... something ... has been implemented/analyzed/etc.` (instead of `we did, we implemented, we analyzed`).\newline In the introduction you should give to the reader enough information to understand what is going to be explained in the remainder of the report (basically, expanding some concept you mentioned in the Abstract) without giving away too many information that would make the introduction too long and boring.\newline Feel free to organize the introduction in multiple sections and subsections, depending on how much content you want to put into this chapter.

Remember that the introduction is needed to make the reader understand what kind of reading he or she will encounter. Be fluent and try not to confuse him or her.
The introduction must ALWAYS end with the following formula: The remainder of the document is organized as follows. In Chapter 2, ...; in Chapter 3, ... so that the reader can choose which chapters are worth skipping according to the type of reading he or she has chosen.
