\chapter{Results}

\subsection{Case of study 1: Cluster and Honeypot for a Ransomware attack}

Results show that the solution is effective since most of the file system is spared from the encryption. \\
Malware honeypots belong to the wider category of research honeypots. The behavior of the malware should be tracked in order to gather information and eventually create better ad-hoc countermeasures. Moreover, the system of insertion of infected nodes into a blacklist stops the spread of the malware, which is one of the crucial points to tackle when dealing with malware and their power of infection.


\subsection{Case of study 2: Cluster and Honeypot for a DoS attack}

\subsubsection{High level version}
We run the DosAttack.sh with different number of scripts launched, 50,100,150,200 and 400. For this version, the cluster with the honeypot was able to handle up to 200 fake sensors. 
Unfortunately with 400 Dos scripts the virtual machine where the system is simulated wasn't able to establish all the sockets, due to the high number of resources requested.

\subsubsection{Low level version} 
We run the DosAttack.sh with different number of scripts launched, 50,100,150,200 and 400. For this version, the cluster with the honeypot inside the dispatcher was able to handle up to 400 fake sensors. 
This version works better than the other because the dispatcher don't have to redirect the data from the Dos sensors to the honeypot, because it is instantiated inside it.\\
Results obtained are two solutions for two different attacks on two complementary clusters, a P2P cluster where therefore all the nodes are the same and talk to each other without hierarchies and a cluster that works in a very hierarchical way, with a main server and all the others that they accept connections from clients through a dispatcher.
Same for the two attacks that are also completely different from each other as we have fully explained in the previous chapters.\\
Testing the DoS solution we noticed that it is able to manage small DoS attacks (this is also due to the virtual machine which obviously does not allow an exhaustive test). If attacked, our solution manages to perfectly manage 100 attackers, without creating delay to the server, reaching up to ...

\section{Known Issues}

\subsection{Case of study 1: Cluster and Honeypot for a Ransomware attack}

\begin{itemize}
    \item The masking of commands is ready and functional, but the commands dispatched by the ransomware cannot fully exploit it. So far the only strategy adopted is considering an encryption procedure capable of encrypting files retrieved in a recursive manner with the 'find' command, thus in alphabetical order. 
    Files we are delivering will include the code that was intended to show the behavior of commands masking procedure. Unfortunately, sending this long command through MQTT will broke it's the execution for some reason that we were not able to solve in a complete way. Specifically the malwere command will recursively list all files applying then the very same encryption mechanism as before. The custom visit to the system directory allows the insertion of the specific masked commands.
    Although this is an issue it is now the main target for future works at the moment of deliverance;
    \item JSON decode function returns a decode error, once in a while, when reading the file "blacklist.json", this only happens rarely.
\end{itemize}

\subsection{Case of study 2: Cluster and Honeypot for a DoS attack}

There are many way to improve our simulation before deploy it on a real hardware implementation. First we need to model more sensor. Then we need to encript and decript the data in the socket, otherwise the attacher could understand the ID of the sensors and invalidate the honeypot. Then we have to create a server that works on multiple thread and that is able to speak with other servers.

\subsubsection{Hardware Limitation}

The Dos solution has suffered a lot from the limited number of raspberry available this because, differently from the solution developed in parallel, it needs two additional raspberries for dispatchers and honeypots that behave differently from all of them, for this reason the solution is limited to a single server , dispatcher, honeypot, and two types of clients.

\section{Future Work}

\subsection{Case of study 1: Cluster and Honeypot for a Ransomware attack}

In a later release of this project we could develop a more realistic ransomware, also considering issues of the actual implementation, or even deploy a real one and release it on a virtualized space to limit damages.\\
We could also shift from a polling monitoring to a smarter one that needs less effort. For instance, we could just monitor the integrity of sentinel files after a command is dispatched. There is also the possibility of wrapping linux system calls directly gaining a more precise way to manage and check sentinel files, in this case the root of trust will be the kernel space of the operating system.

\subsection{Case of study 2: Cluster and Honeypot for a DoS attack}
There are many way to improve our simulation before deploy it on a real hardware implementation. First we need to model more sensor. Then we need to encript and decript the data in the socket, otherwise the attacher could understand the ID of the sensors and invalidate the honeypot. Then we have to create a server that works on multiple thread and that is able to speak with other servers. 